\documentclass[a4paper,11pt,twoside]{article}
\usepackage[utf8]{inputenc}	% Text coding
\usepackage[T1]{fontenc}
\usepackage{lmodern}
\usepackage[czech]{babel}
\usepackage{epsfig}
\usepackage{amsfonts,amsmath,amssymb}
\usepackage{graphicx}
\usepackage[unicode]{hyperref}
\usepackage{indentfirst}
\usepackage{fancyhdr}
\usepackage{xifthen}
\usepackage{amsthm,thmtools}
\usepackage{bold-extra}
\usepackage[dvipsnames]{xcolor}
\usepackage[subrefformat=simple,labelformat=simple]{subcaption} % Instead of subfigure
\usepackage{listings}
%\usepackage{showlabels}

\hypersetup{
	pdftitle={Využití počítačů ve fyzice},
	pdfauthor={Pavel Stránský},
	pdffitwindow=true,
	colorlinks=true,
	urlcolor=cyan,
	linkcolor=red,
	citecolor=green,
	filecolor=magenta
}

\graphicspath{{figures/}}

\lstdefinestyle{listingsstyle}{
    basicstyle=\ttfamily,
    breakatwhitespace=false,         
    breaklines=true,                 
    keepspaces=true,                 
    showspaces=false,                
    showstringspaces=false,
    showtabs=false,                  
    tabsize=2,
    columns=fullflexible
}

\lstset{style=listingsstyle}

%\babelhyphenation[czech]{ma-te-ma-tic-kých}

\renewcommand\thesubfigure{(\alph{subfigure})}

% Page size
\addtolength{\topmargin}{-1.5cm} %\addtolength{\textheight}{-10cm}
\addtolength{\textwidth}{4cm} \addtolength{\textheight}{4cm} % Width and height of the text
\addtolength{\voffset}{-0.5cm} % Top margin
\addtolength{\hoffset}{-2cm}
\setlength{\headheight}{15pt}

\pagestyle{fancy}

\DeclareMathOperator{\e}{e}

\def\vector#1{\boldsymbol{#1}}								% Vector
\renewcommand{\d}{\mathrm{d}}
\newcommand{\derivative}[3][]{\ifthenelse{\isempty{#1}}	    % Normal derivative
	{\frac{\d{#2}}{\d{#3}}}
	{\frac{\d^{#1}{#2}}{\d{#3}^{#1}}}
}
\newcommand{\im}{\mathrm{i}}

\def\makematrix#1{\begin{pmatrix}#1\end{pmatrix}}       % Matrix
\def\abs#1{\left|#1\right|}
\def\probability#1{\mathrm{Pr}\left[#1\right]}
\def\expectation#1{\mathrm{E}\left[#1\right]}
\def\dispersion#1{\sigma_{#1}^{2}}

\def\code#1{\textnormal{\texttt{#1}}}
\def\file#1{\textnormal{\textbf{\texttt{#1}}}}

\def\abbreviation#1{\textnormal{\textsc{#1}}}

\long\def\python#1{{\color{ForestGreen}#1}}

\newtheoremstyle{spaced}
{5pt}{5pt}{\itshape}{}{\bfseries}{:}{.5em}{}

\newtheoremstyle{red}
{5pt}{5pt}{\itshape\color{red}}{}{\bfseries\color{red}}{:}{.5em}{}

\newtheoremstyle{blue}
{5pt}{5pt}{\itshape\color{blue}}{}{\bfseries\color{blue}}{:}{.5em}{}

\begin{document}

\theoremstyle{spaced}
\newtheorem{example}{Příklad}[section]

\theoremstyle{red}
\newtheorem{task}{Úkol}[section]

\theoremstyle{blue}
\newtheorem{solution}{Řešení}[section]

\title{Zápisky k předmětu Využití počítačů ve fyzice}
\date{\today}
\author{Pavel Stránský}

\maketitle
\tableofcontents

\section{Instalace používaných nástrojů}
\label{sec:Instalace}
    Příklady k cvičení budou demonstrovány v nejnovější verzi programovacího jazyka \href{https://python.org}{Python}.
    Jako vývojové prostředí doporučuji \href{https://code.visualstudio.com}{Visual Studio Code}.
    Tento volně dostupný program lze nainstalovat na všechny neužívanější operační systémy (Linux, Windows, macOS).    
    Má nepřeberné možnosti při editaci zdrojových souborů, překladu a ladění snad ve všech známých programovacích jazycích.
    Bohaté možnosti nastavení umožnují přizpůsobit si práci svým potřebám (například zvýrazňování syntaxe, klávesové zkratky či vzhled prostředí).
    Pomocí doplňků ho můžete integrovat s dalšími službami, například s verzovacím programem Git, či vzdálenými repozitáři, čehož také využijeme.
    Komunita, která toto vývojové prostředí používá a spravuje, je obrovská, což zaručuje dobrou podporu a rychlé přidávání nových funkcí.

\subsection{Instalace Pythonu}
\label{sec:Python}
    Instalační soubor pro svůj operační systém stáhnete ze stránky \href{https://python.org}{python.org}.
    Při instalaci na počítač s Windows doporučuji zvolit \uv{Add Python to PATH}, což zjednoduší práci s Pythonem z příkazové řádky, a na poslední obrazovce zvolit \uv{Disable path length limit}:
    \begin{center}
        \includegraphics[width=0.495\linewidth]{PythonInstallPath.png}
        \includegraphics[width=0.495\linewidth]{PythonInstallPathLimit.png}
    \end{center}
    Pro ověření instalace napíšeme v příkazové řádce příkaz \code{python}.\footnote{Na počítačích s Linuxem je příkaz v terminálu \code{python3}.}
    Tím se spustí \abbreviation{REPL}\footnote{{\bf R}ead-{\bf E}valuate-{\bf P}rint-{\bf L}oop.} Pythonu, ve které můžeme již psát všechny příkazy programovacího jazyka, které se po zadání ihned provedou a vypíšou výsledek.

\subsubsection{Instalace doplňujících knihoven}
    Samotná instalace Pythonu obsahuje jen minimální množství nejnutnějších knihoven.
    My budeme využívat ještě následující rozšiřující knihovny:
    \begin{itemize}
        \item \file{\href{https://numpy.org}{NumPy}} ({\bf Num}erical {\bf Py}thon: numerická matematika, řady a vícedimenzionální datové typy),
        \item \file{\href{https://scipy.org}{SciPy}} ({\bf Sci}entific {\bf Py}thon: algoritmy pro optimalizaci, statistiku, řešení diferenciálních rovnic, lineární algebru, atd.),
        \item \file{\href{https://matplotlib.org}{Matplotlib}} (vizualizace, grafy).
    \end{itemize}
    K jejich doinstalování slouží modul \code{pip}.
    V příkazové řádce napíšeme
    \begin{lstlisting}
        python -m pip install numpy scipy matplotlib
    \end{lstlisting}
    čímž se nainstalují naráz všechny tři knihovny.\footnote{
        Pokud na počítači s Linuxem uvedený postup nebude fungovat, je potřeba nejprve nainstalovat instalátor \file{pip} pomocí příkazu \code{sudo apt install python3-pip}.
        Pak lze použít buď výše uvedený příkaz, nebo stručnější \code{pip3 install numpy}.
    }
    
    Existuje samozřejmě celá řada dalších užitečných a používaných knihoven, jako je například \file{\href{https://pandas.pydata.org/}{Pandas}} pro analýzu dat nebo \file{\href{https://www.sympy.org/}{SymPy}} pro symbolické výpočty, na které v tomto kurzu nedojde.

\subsection{Instalace Visual Studio Code}
    Instalace jazyka Python obsahuje jednoduché vývojové prostředí nazvané \abbreviation{\href{https://docs.python.org/3/library/idle.html}{IDLE}}.\footnote{
        {\bf I}ntegrated {\bf D}evelopment and {\bf L}earning {\bf E}nvironment.
    }
    To však poskytuje jen omezené možnosti co se týče ladění, psaní rozsáhlejších projektů s více zdrojovými soubory nebo integrace s verzovacími programy.

    Pro serióznější práci budeme používat \href{https://code.visualstudio.com}{Visual Studio Code} s doplňkem pro programovací jazyk Python.
    Instalační soubor stáhnete ze stránky \href{https://code.visualstudio.com}{code.visualstudio.com}.
    Během instalace na počítač s Windows doporučuji zvolit obě možnosti \uv{Add Open with Code action to...},
    \begin{center}\includegraphics[width=0.5\linewidth]{VSCodeInstall.png}\end{center}
    což zjednoduší otevírání složek s projekty nebo samostatných souborů pomocí pravého tlačítka myši.

    Na operačním systému Linux je nejjednodušší provést instalci pomocí Snap Store příkazem v terminálu
    \begin{lstlisting}
        sudo snap install --classic code
    \end{lstlisting}
    nebo použít tento \href{https://code.visualstudio.com/docs/setup/linux}{návod}.

\subsubsection{Instalace doplňku pro Python}
    \begin{center}\includegraphics[width=\linewidth]{VSCodeInstallPython.png}\end{center}
    Ke správě doplňků (extensions) se dostanete kliknutím na ikonku {\color{red} 1} nebo stisknutím \code{Ctrl+Shift+X}. 
    Vyhledáte doplněk \href{https://marketplace.visualstudio.com/items?itemName=ms-python.python}{Python} {\color{red} 2} od Microsoftu, vyberete ho {\color{red} 3} a nainstalujete {\color{red} 4}.

\subsection{Instalace Git}
    Instalační soubor verzovacího systému \href{Git}{https://git-scm.com} stáhnete z webové stránky \href{https://git-scm.com}{git-scm.com}.
    K instalaci Gitu potřebujete administrátorská práva.
    
    V následujícím postupu zobrazuji jen ty snímky obrazovky, na kterých je vhodné vybrat jinou volbu, než jaká je instalátorem standardně nabízena.
    \begin{center}\includegraphics[width=0.5\linewidth]{GitInstallEditor.png}\end{center}
    Jako editor zvolíme dříve nainstalované Visual Studio Code.
    Systém Git vyžaduje editor jednak pro povinný komentář každé zapsané změny (commit), jednak pro ošetření kolizí při slučování větví (merge).
    \begin{center}\includegraphics[width=0.5\linewidth]{GitInstallMain.png}\end{center}
    Hlavní větev repozitáře se donedávna standardně jmenovala \code{master}.
    Vzhledem k negativním konotacím tohoto slova v angličtině se přechází na neutrálnější označení.
    Nejběžnější je \code{main}, na které již přešel i populární správce vzdálených repozitářů \href{https://github.com}{GitHub}.
    Doporučuji tedy zvolit pojmenování \code{main}.

    Všechna nastavení lze samozřejmě kdykoliv po instalaci změnit.

\section{Úvod do systému Git}
    Git je verzovací systém.
    Pomáhá udržet historii změn souborů projektu, přičemž ke každému snímku historie se lze vrátit, a pokud se jedná o textový soubor, umí porovnat aktuální a historickou verzi řádek po řádku a zvýraznit provedené změny.    
    Git umožňuje pracovat s nezávislými vývojovými větvemi (\emph{branches}), ve kterých lze například zkoušet různý přístup k řešení daného problému a mezi kterými lze jednoduše přepínat.
    Nejlepší řešení lze pak snadno začlenit (\emph{merge}) do hlavní vývojové větve.

    Git uchovává nejen historické verze souborů, ale také informace o tom, kdo změny provedl.
    Proto se používá k práci v týmu, kdy každý člen týmu pracuje na určité části projektu a své změny následně do projektu včleňuje.
    Pod správou Gitu se snadno vyvíjejí i velké projekty, například \href{https://github.com/microsoft/vscode}{Visual Studio Code}.
    Tento projekt je otevřený (open source), což znamená, že kdokoliv, tedy i vy nebo já, má přístup ke všem zdrojovým kódům, může se do práce na projektu zapojit a přispět k jeho vývoji.
    
    Soubory projektu a informace o jejich historických změnách se nazývá souhrnně \emph{repozitář}.
    Git patří mezi distribuované verzovací systémy.
    To znamená, že každý uživatel má na svém počítači celý obsah repozitáře.
    Hlavní výhody tohoto přístupu oproti centrálně řízeným verzovacím systémům jsou dvě:
    \begin{enumerate}
        \item Práce na projektu nevyžaduje připojení k centrálnímu serveru s repozitářem, a tedy můžete pracovat offline klidně někdě v džungli nebo na Marsu.
        \item Případná porucha počítače spojená se ztrátou dat není žádná katastrofa, protože ostatní členové týmu mají identickou kopii repozitáře na svých počítačích.
    \end{enumerate}
    
    Git nejefektivněji funguje na textové soubory, ale zvládne verzovat i soubory binární (například obrázky).

    Program Git pochází z prostředí Linuxu, a proto je navržený tak, aby se s ním dalo pracovat z příkazové řádky (terminálu) pomocí jednoduchých textových příkazů.
    Pro snadnější a rychlejší práci s Gitem je pak dispozici bezpočet různých grafických nadstaveb.  
    Práce s Gitem je dokonce integrována přimo do vývojového prostředí Visual Studio Code.

    Na stránkách projektu Git najdete \href{https://git-scm.com/book/en/v2}{podrobný interaktivní návod} ke všem funkcím verzovacího systému (a to částečně i v \href{https://git-scm.com/book/cs/v2}{češtině}).

\subsection{Prvotní nastavení}
    Git nelze používat do té doby, než jsou nastaveny základní informace o uživateli.
    To se provede pomocí příkazů v příkazové řádce (terminálu)
    \begin{lstlisting}
        git config --global user.name "..."
        git config --global user.email "..."
    \end{lstlisting}
    přičemž za ... doplníte své jméno (přezdívku) a email.
    Těmito údaji se podepisují všechny zapsané změny v repozitáři.
    Pokud tedy spolupracujete na projektu s jinými lidmi, je dobré zvolit takové údaje, pomocí kterých vás kolegové dobře identifikují a snadno kontaktují.

    Výpis všech nastavení získáte příkazem
    \begin{lstlisting}
        git config --global
    \end{lstlisting}
    Uvidíte, že v mezi nastaveními je i název hlavní větve repozitáře (\code{init.defaultbranch=main}) a cesta k nastavenému textovému editoru (v našem případě Visual Studio Code).

    Všechna nastavení a práci s příkazem \code{config} najdete v tomto \href{https://git-scm.com/docs/git-config}{podrobném návodu} nebo zadáním příkazu
    \begin{lstlisting}
        git help config
    \end{lstlisting}

\subsection{Cheat Sheet}
    Pro snadnou orientaci v použití Gitu si můžete vytisknout \href{https://training.github.com/downloads/github-git-cheat-sheet.pdf}{Git Cheat Sheet} nebo využít \href{https://ndpsoftware.com/git-cheatsheet.html}{tento interaktivní web}.

\section{Úvod do Pythonu}
    Python je v dnešní době velmi populární programovací jazyk, který pronikl do spousty nesouvisejících odvětví, a to zejména díky bohatosti a pokročilosti knihoven, které jsou vyvíjeny komunitou a jsou tudíž aktuální a dobře odladěné.
    Python existuje na mnoha platformách od osobních počítačů po mikrokontroléry a dá se v něm naprogramovat téměř cokoliv: pokročilé vědecké výpočty používající nejnovější numerické algoritmy, dobře vypadající grafy, statistická analýza, analýza velkých dat a strojové učení, ale také webové služby, okénkové programy či editace obrázků a videí.
    Je běžné, že i komerční programy obsahují Python coby skriptovací jazyk a tím umožňují uživateli používat své vlastní kódy \uv{uvnitř} komerčního produktu.\footnote{
        Jako příklad poslouží nejnovější verze programu \href{https://www.wolfram.com/language/12/external-system-integration/evaluate-python-in-a-notebook.html}{Mathematica}, \href{https://www.originlab.com/doc/python/Run-Python-in-Origin}{Origin} či \href{https://developer.sas.com/guides/python.html}{SAS}.
    }
    
    Python lze charakterizovat jako jazyk:
    \begin{itemize}
        \item 
            \emph{Interpetovaný}, což znamená, že provádění programů probíhá přímo ze zdrojového kódu.\footnote{
                Mezi další známé interpetované jazyky patří napříkad JavaScript nebo PHP.
            }
            Ke spouštění kódů je tedy potřeba mít nainstalovaný interpret, což jsme učinili v sekci~\ref{sec:Python}.
            Pokud kód obsahuje syntaktickou chybu, je odhalena až ve chvíli, kdy se na ni při provádění kódu narazí.
            U interpretovaných jazyků se neztrácí čas překladem do strojového kódu a je pro ně přirozené dynamické typování proměnných.
            Tím, že interpet čte text kódu příkaz po příkazu, je provádění programů pomalejší, ale vzhledem k tomu, že velká část časově náročných funkcí a knihoven bývá napsána v rychlejších programovacích jazycích, není to zásadní nevýhoda.
            
            Opakem intepretovaných jazyků jsou jazyky kompilované, a to buď přímo do strojového kódu daného procesoru\footnote{Například C/C++, Pascal nebo Fortran.} nebo do mezikódu, který ke spuštění vyžaduje dodatečnou kompilaci, která však může být vysoce optimalizovaná přímo pro danou hardwarovou konfiguraci použitého počítače.\footnote{Zde se jedná napřílad o jazyky Java, C\# a celý .NET framework nebo Julia.}

        \item 
            \emph{Dynamicky typovaný}, což znamená, že proměnná nemusí mít předem daný typ a typ proměnné se může za běhu programu dokonce měnit.
        
        \item 
            \emph{S automatickou správou paměti}, takže programátor se nemusí starat o alokování a uvolňování paměti. 
            Python při inicializaci proměnné paměť automaticky alokuje a ve chvíli, kdy proměnnou přestaneme používat, paměť uvolní.\footnote{Algoritmus se nazývá \emph{Garbage collection}.} 

        \item
            V Pythonu lze svým spůsobem používat tři základní programovací paradigmata: \emph{procedurální, objektové a funkcionální}.
    \end{itemize}
         
    Python klade důraz na jednoduchost, stručnost a čitelnost kódu.
    Filosofie programovacího jazyka je shrnuta v \href{https://www.python.org/dev/peps/pep-0020/}{Zenu Pythonu}.\footnote{Zen se také vypíše, pokud do svého kódu zadáte \code{import this}.}

\subsection{Vytvoření a spuštění kódu}
    Ve Visual Studio Code vytvoříme nový soubor a uložíme si ho s příponou \code{*.py}. 
    Podle přípony totiž VS Code pozná, že se jedná o pythonovský kód a použije k práci s ním správný doplněk.
       
    Hotový kód spustíte ve VS Code jedním z následujících tří způsobů.
    \begin{enumerate}
        \item 
            \emph{Kliknutím na zelenou šipku na nástrojové liště}:
            Tímto způsobem spustíte celý soubor kódu.
        \item
            \emph{Stiskem \code{F5}} (a výběrem Python File): Kód se spustí v režimu ladění (debugger).
            Zastaví se na každém kontrolním bodě (breakpoint, vloží se na vybranou řádku kódu klávesou \code{F9}) nebo na každé chybě.
            Pro pokračování provádění kódu stiskněte buď znovu \code{F5}, nebo \code{F10} pro jeden krok.

        \item
            \emph{Označením části kódu a stiskem Ctrl+Enter}: V okně označeném \code{TERMINAL} spustí \abbreviation{REPL} Pythonu a v něm označenou část kódu.
            \abbreviation{REPL} se po provedení kódu nezavře, takže v něm lze psát dodatečné příkazy.
            Pro ukončení \abbreviation{REPL} do něj stačí napsat \code{exit()} nebo v něm stisknout \code{Ctrl+Z} a potvrdit.
    \end{enumerate}

\subsection{Základy syntaxe Pythonu}
    Soubor se vzorovými příklady najdete v repozitáři ke cvičení: \file{\href{https://github.com/PavelStransky/PCInPhysics2021/blob/main/python/PythonBasics.py}{PythonBasics.py}}.\footnote{
        Vzorový soubor je inspirován příkladem \href{https://learnxinyminutes.com/docs/cs-cz/python/}{Nauč se Python v Y minutách}.
    }
    Doporučuji vám si ho stáhnout nebo jeho obsah překopírovat a důsledně si ho řádek po řádku projít a spouštět nejlépe pomocí označení a stiskem \code{Ctrl+Enter}, jak bylo popsáno v předchozí sekci.
    Některé části kódu odkazují na proměnné zavedené v dřívější části kódu, proto kód procházejte postupně od začátku do konce.

    Z vzorového souboru bych vypíchl následující body:
    \begin{enumerate}
        \item 
            \emph{Základní datové typy a operátory}: Zaměřte se na možnosti formátování řetězců. 
            Python obsahuje jen dva základní číselné typy: \code{int} a \code{float}.
            Zbytek je podobný jako v jiných programovacích jazycích.

        \item
            \emph{Proměnné a kolekce}: 
            Důležité standardní kolekce jsou \emph{seznam} (list) [...] a \emph{slovník} (dicctionary) \{...\}.
            Dále existuje typ \emph{n-tice} (tuple) (...) a \emph{množina} (set) \{...\}.
            O jaký typ kolekce se jedná poznáte podle typu závorek, kterými je uzavřena.
            Python nabízí bohaté možnosti indexování prvků kolekcí.

            Python nemá typ \uv{řada} (jedno či vícerozměrný soubor hodnot stejných typů).
            Ten je implementován až v rozšiřujících knihovnách, například v knihovně \file{numpy}, které se budeme věnovat později.
            
        \item
            \emph{Podmínky a cykly}:
            Příkaz uvozující blok kódu vždy končí znakem :.
            Každý blok je definován svým odsazením, které musí být na každém řádku stejné
            (tj. musí obsahovat stejný počet odsazujících znaků, mezi které patří buď mezera nebo tabulátor).
            Konvence je používat k odsazení \href{https://www.python.org/dev/peps/pep-0008/#indentation}{4 mezery}.

            Cykly jsou možné pouze přes iterovatelné objekty.
            Pro cyklus přes přirozená čísla (indexy) musíme vytvořit odpovídající iterovatelný objekt příkazem \code{range}.
            V drtivé většině se lze při programování v Pythonu indexům zcela vyhnout.

        \item 
            \emph{Funkce}: Python umožnuje velkou variabilitu co se týče argumentů funkce a návratových hodnot díky své funkci automatického sbalení a rozbalení kolekcí.
            Funkce se navíc chová jako objekt, lze ji tedy přiřadit jakékoliv proměnné.
            To je jeden z prvků funkcionálního programování.

            Funkcionální programování také obsahuje koncept anonymní funkce, což je jednoduchá funkce definovaná na jednom řádku, která nemá vlastní jméno.
            V Pythonu se vytváří pomocí klíčového slova \code{lambda}.

        \item 
            \emph{Moduly}: Při programování je důležité vhodně strukturovat kód.
            Python obsahuje jednoduchý koncept modulů, přičemž každý soubor s příponou \code{*.py} lze použít jako modul.
            Modul je vlastně speciální typ objektu.

            Dobře se seznamte se způsoby, jak modul načíst a používat, jelikož většina funkcí Pythonu se nachází právě v modulech.
            Jedná se například o matematické funkce v modulu \file{math}.

        \item 
            \emph{Třídy}: Python umožňuje objektově orientované programování.
            Práce s třídami se však v mnohém liší od striktně objektově orientovaných programovacích jazyků, jakými jsou například C++, C\# nebo Java.
            Důležitý rozdíl je například v přístupnosti atributů (všechny atributy v Pythonu jsou veřejné).
            Rovněž dědičnost a dědění metod se chová odlišně.
            Dále je nutné pamatovat na to, že každá metoda třídy musí mít v deklaraci jako první argument odkaz na instanci, se kterou je volána (konvenčně se označuje \code{self}).
            Ve vzorovém souboru je opravdu jen to nejnutnější minimum.
            Pokud chcete v Pythonu využívat objekty seriózně, doporučuji pročíst si odpovídající \href{https://docs.python.org/3/tutorial/classes.html}{kapitolu v manuálu}.
    \end{enumerate}

    Uvedený soubor s příklady obsahuje jen ty struktury jazyka Python, které budeme používat.
    V budoucích cvičeních se ještě seznámíte se \emph{správou kontextu} (klíčové slovo \code{with}).
    Python umožňuje navíc 
    \begin{itemize}
        \item ošetření chyb pomocí \emph{výjimek} (klíčová slova \code{raise}, \code{try}, \code{except}, \code{finally}),
        \item tvorbu \emph{generátorů} (klíčové slovo \code{yield}),
        \item \emph{asynchronní programování}, korutiny a úkoly (klíčová slova \code{await}, \code{async})
        \item či tvorbu a použití \emph{dekorátorů}.
    \end{itemize}
    Pro plné ovládnutí jazyka vám doporučuji se v budoucnu s těmito koncepty seznámit, a to buď ze specializovaných přednášek, z tutoriálů a návodů na webu, nebo studiem cizích kódů.

\subsection{Pojmenování proměnných a formátování}
    Formátování kódu Pythonu je celkem volné.
    Povinné je dodržet jen správné odsazení bloků.
    Pro snazší čitelnost kódu byla nicméně vytvořena řada doporučení, která najdete souhrnně na stránce \href{https://www.python.org/dev/peps/pep-0008}{PEP 8}.\footnote{\abbreviation{PEP} = {\bf P}ython {\bf E}nhancement {\bf P}roposal}
        
    Pro pojmenování proměnných existuje jediné závazné pravidlo, které zní, že indentifikátor se nesmí shodovat s žádným z \href{https://docs.python.org/3/reference/lexical_analysis.html#keywords}{35 klíčových slov} jazyka.
    Kromě toho jsou ve výše uvedeném dokumentu další nezávazná pravidla k pojmenování proměnných:
    \begin{itemize}
        \item
            \emph{Proměnné a funkce} se doporučuje pojmenovávat malými písmeny a jednotlivá slova spojovat podtržítky, tzv. underscore style (například \code{pocet\_bodu}).
        \item
            \emph{Konstanty} se doporučuje pojmenovávat velkými písmeny a jednotlivá slova spojovat podtržítky (například \code{PLANCKOVA\_KONSTANTA}).
        \item
            \emph{Třídy} se doporučuje pojmenovávat tak, že jednotlivá slova názvu začínají velkým písmenem a mezi nimi není žádný znak, tzv. Pascal style (například \code{PostovniAdresa}).
        \item
            Kvůli čitelnosti je dobré se vyhnout jednopísmenným označením \code{l}, \code{I} a \code{O}, jelikož takto označené proměnné mohou být snadno zaměněny s čísly \code{1} a \code{0}.

    \end{itemize}
    Pokud vám vyhovuje jiný styl pojmenovávání, lze ho použít.
    Je však dobré být v pojmenovávání konzistentní napříč celým projektem.

    Častá otázka je, zda proměnné označovat \emph{česky či anglicky}.
    Jelikož nikdy nevíte, kdo v dnešním propojeném světě bude chtít váš kód použít a třeba i upravit pro své potřeby, je vhodnější používat anglická pojmenování.
    
    Snadná čitelnost kódu je podpořena i vhodným psaním \emph{komentářů}.
    Obecné tvrzení zní, že dobře napsaný kód je čitelný i bez komentářů.
    Poslouží hlavně vhodně označené proměnné a správně navržená struktura kódu.
    Komentovat je dobré jen ty části kódu, kde se používá nějaký ne všeobecně známý trik nebo postup.
    Nadbytek komentářů je velmi těžké udržovat při změnách kódu a může vést i k tomu, že po několika úpravách nebudou komentáře v souladu s tím, co kód vykonává.
    
    Naopak dobré je nešetřit vysvětlením, co dělají jednotlivé funkce, jaké jsou jejich argumenty a co vracejí na výstupu.
    K tomu slouží komentář typu \emph{docstring}, který je vysvětlený v souboru \file{\href{https://github.com/PavelStransky/PCInPhysics2021/blob/main/python/PythonBasics.py}{PythonBasics.py}} v sekci o funkcích.
    Použití \emph{docstringu} usnadní i tvorbu a správu doprovodné dokumentace k celému projektu.
    Existují například nástroje, které vám z těchto komentářů vytvoří strukturované webové stránky.

\section{Obyčejné diferenciální rovnice}
\subsection{Diferenciální rovnice prvního řádu}
    Nejprve se budeme věnovat řešení jedné diferenciální rovnice prvního řádu,
    \begin{equation}
        \derivative{y}{t}=f(y,t)
    \end{equation}
    s počáteční podmínkou
    \begin{equation}
        y(t_{0})=y_{0}.
    \end{equation}
    Zde $y=y(t)$ je hledaná funkce a $t$ je nezávisle proměnná.
    
    Numerické řešení diferenciální rovnice spočívá v nahrazení infinitezimálních přírůstků přírůstky konečnými:
    \begin{equation}\label{eq:Difference}
        \frac{\Delta y}{\Delta t}=\phi(y,t)
    \end{equation}
    kde $\phi$ je funkce, která udává směr, podél kterého se při numerickém řešení vydáme.
    Volbá této funkce je klíčová a záleží na ní, jak přesné řešení dostaneme a jak rychle ho dostaneme.

    \subsubsection{Pár důležitých pojmů}
    \begin{itemize}
        \item{\bf Explicitní algoritmy:}
        K výpočtu hodnoty funkce $y_{i+1}$ se vyžadují pouze hodnoty z aktuálních a minulých kroků, tj. $y_{i}$, $y_{i-1}$, atd.

        \item {\bf Jednokrokové algoritmy:}
        K výpočtu hodnoty funkce $y_{i+1}$ v následujícím kroku vyžadují pouze znalost hodnoty funkce v aktuálním kroku $y_{i}$.
        Rozepsáním~\eqref{eq:Difference} dostaneme
        \begin{equation}
            \boxed{
                y_{i+1}=y_{i}+\underbrace{\phi(y_{i},t)}_{\phi_{i}}\Delta t
            },
        \end{equation}
        přičemž počáteční hodnota $y_{0}$ je dána počáteční podmínkou.
        My se omezíme pouze na tyto algoritmy.

        \item {\bf Lokální diskretizační chyba:}
        \begin{equation}
            \mathcal{L}=y(t+\Delta t)-y(t)-\phi(y(t),t)\Delta t,
        \end{equation}        
        kde $y(t)$ udává přesné řešení v čase $t$.

        \item {\bf Akumulovaná diskretizační chyba:}
        \begin{equation}
            \epsilon_{i}=y_{i}-y(t_{i})
        \end{equation}

        \item {\bf Řád metody:} 
        Metoda je $p$-tého řádu, pokud
        \begin{equation}\label{eq:MethodOrder}
            L(\Delta t)=\mathcal{O}(\Delta t^{p+1}).
        \end{equation}

        \item {\bf Kontrola chyby řešení:}
        Chybu numerického řešení diferenciální rovnice lze zmenšit 1)~menším krokem, 2)~lepší metodou (metodou vyššího řádu). 
        Menší krok však znamená vyšší výpočetní čas.
        Sofistikované metody proto průběžně mění velikost kroku: když se funkce mění pomalu, krok prodlouží, když se mění rychle, krok zkrátí (tzv. {\bf metody s adaptivním krokem}).
        Tím se docílí vysoké přesnosti při co nejmenším výpočetním čase.

    \end{itemize}

    \subsubsection{Eulerova metoda 1. řádu}
        \begin{equation}\label{eq:Euler1}
            \phi_{i}=f(y_{i},t_{i}),
        \end{equation}
        tj. krok do $y_{i+1}$ děláme vždy ve směru tečny v bodě $y_{i}$.

        \begin{itemize}
            \item Nejjednodušší metoda integrace diferenciálních rovnic.
            \item Chyba je obrovská, k dosažení přesných hodnot je potřeba velmi malého kroku, což znamená dlouhý výpočetní čas.
        \end{itemize}

    \subsubsection{Eulerova metoda 2. řádu}
        \begin{align}\label{eq:Euler2a}
            k_{1}&=f(y_{i},t_{i})\nonumber\\
            k_{2}&=f\left(y_{i}+k_{1}\Delta t,t+\Delta t\right)\\
            \phi_{i}&=\frac{1}{2}\left(k_{1}+k_{2}\right),\nonumber
        \end{align}
        tj. uděláme jednoduchý Eulerův krok ve směru $k_{1}$, spočítáme derivaci $k_{2}$ po tomto kroku a vyrazíme z bodu $y_{i}$ ve směru, který je průměrem obou směrů (doporučuji si nakreslit obrázek).

        Ekvivalentní je udělat \uv{Eulerův půlkrok} a vyrazit z bodu $y_{i}$ ve směru derivace spočtené po~tomto půlkroku:
        \begin{align}\label{eq:Euler2b}
            k'_{1}&=f(y_{i},t_{i})\nonumber\\
            k'_{2}&=f\left(y_{i}+k'_{1}\frac{\Delta t}{2},t+\frac{\Delta t}{2}\right)\\
            \phi_{i}&=k'_{2}\nonumber
        \end{align}

    \subsection{Runge-Kuttova metoda 4. řádu}
        \begin{align}\label{eq:RungeKutta}
            k_{1}&=f(y_{i},t_{i})\nonumber\\
            k_{2}&=f\left(y_{i}+k_{1}\frac{\Delta t}{2},t+\frac{\Delta t}{2}\right)\nonumber\\
            k_{3}&=f\left(y_{i}+k_{2}\frac{\Delta t}{2},t+\frac{\Delta t}{2}\right)\\
            k_{4}&=f\left(y_{i}+k_{3}\Delta t,t+\Delta t\right)\nonumber\\
            \phi_{i}&=\frac{1}{6}\left(k_{1}+2k_{2}+2k_{3}+k_{4}\right)\nonumber
        \end{align}
        
        \begin{itemize}
            \item Jedna z nejčastěji používaných metod.
            \item Vysoká rychlost a přesnost při relativní jednoduchosti.
            \item Existují i Runge-Kuttovy metody vyššího řádu $p$, avšak vyžadují výpočet více než $p$ dílčích derivací $k_{j}$.
            Obecně platí, že metoda řádu $p\leq4$ vyžaduje $p$ derivací, metoda řádu $5\leq p\leq7$ vyžaduje $p+1$ derivací a metoda řádu $p=8,9$ vyřaduje $p+2$ derivací.
        \end{itemize}

    \begin{task}
        Naprogramujte Eulerovu metodu 1. a 2. řádu a Runge-Kuttovu metodu.
        Vyřešte diferenciální relaxační rovnici
        \begin{equation}\label{eq:Exp}
            \derivative{y}{t}=-y
        \end{equation}
        s počátečními podmínkami $y_{0}=1$ (analytickým řešením je funkce $e^{-t}$).
        Integrační krok $\Delta t$ ponechte jako volný parametr.
        Nakreslete grafy řešení $y(t)$ pro rozdílné hodnoty integračních kroků, například $\Delta t=0.01$ a $\Delta t=0.1$ pro čas $t\in\langle0;10\rangle$.
    \end{task}

    \begin{task}
        Rozšiřte kód tak, aby počítal průměrnou kumulovanou chybu
        \begin{equation}\label{eq:ExpError}
            \mathcal{E}=\sqrt{\frac{1}{n}\sum_{i=0}^{n-1}\left(y_{i}-e^{-t_{i}}\right)^{2}}
        \end{equation}
        a nakreslete závislost $\mathcal{E}(\Delta t)$ pro $\Delta t\in\langle0.002;0.1\rangle$ a pro různé metody.
        Jelikož očekáváme mocninnou závislost dle~\eqref{eq:MethodOrder}, kde exponent je tím větší, čím větší je řád metody, je výhodné graf $\mathcal{E}(\Delta t)$ kreslit v log-log měřítku.
        V Pythonu použijete místo \textnormal{\texttt{plot(...)}} funkci \textnormal{\texttt{loglog(...)}} z knihovny \textnormal{\texttt{matplotlib.pyplot}}.
        Ověřte, že získané křivky jsou v souladu s řády použitých metod.
    \end{task}

    \begin{task}
        Pomocí naprogramovaných metod vyřešte nelineární diferenciální rovnici
        \begin{equation}
            \derivative{y}{t}=\sin\left(t y\right)
        \end{equation}
        s počáteční podmínkou $y_{0}=1$ a vykreslete graf jejího řešení.
    \end{task}

\end{document}
