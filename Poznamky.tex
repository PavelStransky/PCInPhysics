\documentclass[a4paper,11pt,twoside]{article}
\usepackage[utf8]{inputenc}	% Text coding
\usepackage[T1]{fontenc}
\usepackage{lmodern}
\usepackage[czech]{babel}
\usepackage{epsfig}
\usepackage{amsfonts,amsmath,amssymb}
\usepackage{graphicx}
\usepackage[unicode]{hyperref}
\usepackage{indentfirst}
\usepackage{fancyhdr}
\usepackage{xifthen}
\usepackage{amsthm,thmtools}
\usepackage{bold-extra}
\usepackage[dvipsnames]{xcolor}
\usepackage[subrefformat=simple,labelformat=simple]{subcaption} % Instead of subfigure
%\usepackage{showlabels}

\hypersetup{
	pdftitle={Využití počítačů ve fyzice},
	pdfauthor={Pavel Stránský},
	pdffitwindow=true,
	colorlinks=true,
	urlcolor=cyan,
	linkcolor=red,
	citecolor=green,
	filecolor=magenta
}

\graphicspath{{figures/}}

%\babelhyphenation[czech]{ma-te-ma-tic-kých}

\renewcommand\thesubfigure{(\alph{subfigure})}

% Page size
\addtolength{\topmargin}{-1.5cm} %\addtolength{\textheight}{-10cm}
\addtolength{\textwidth}{4cm} \addtolength{\textheight}{4cm} % Width and height of the text
\addtolength{\voffset}{-0.5cm} % Top margin
\addtolength{\hoffset}{-2cm}
\setlength{\headheight}{15pt}

\pagestyle{fancy}

\DeclareMathOperator{\e}{e}

\def\vector#1{\boldsymbol{#1}}								% Vector
\renewcommand{\d}{\mathrm{d}}
\newcommand{\derivative}[3][]{\ifthenelse{\isempty{#1}}	    % Normal derivative
	{\frac{\d{#2}}{\d{#3}}}
	{\frac{\d^{#1}{#2}}{\d{#3}^{#1}}}
}
\newcommand{\im}{\mathrm{i}}

\def\makematrix#1{\begin{pmatrix}#1\end{pmatrix}}       % Matrix
\def\abs#1{\left|#1\right|}
\def\probability#1{\mathrm{Pr}\left[#1\right]}
\def\expectation#1{\mathrm{E}\left[#1\right]}
\def\dispersion#1{\sigma_{#1}^{2}}

\def\code#1{\textnormal{\texttt{#1}}}
\def\file#1{\textnormal{\textbf{\texttt{#1}}}}
\long\def\python#1{{\color{ForestGreen}#1}}

\newtheoremstyle{spaced}
{5pt}{5pt}{\itshape}{}{\bfseries}{:}{.5em}{}

\newtheoremstyle{red}
{5pt}{5pt}{\itshape\color{red}}{}{\bfseries\color{red}}{:}{.5em}{}

\newtheoremstyle{blue}
{5pt}{5pt}{\itshape\color{blue}}{}{\bfseries\color{blue}}{:}{.5em}{}

\begin{document}

\theoremstyle{spaced}
\newtheorem{example}{Příklad}[section]

\theoremstyle{red}
\newtheorem{task}{Úkol}[section]

\theoremstyle{blue}
\newtheorem{solution}{Řešení}[section]

\title{Zápisky k předmětu Využití počítačů ve fyzice}
\date{\today}
\author{Pavel Stránský}

\maketitle
\tableofcontents

\section{Obyčejné diferenciální rovnice}
\subsection{Diferenciální rovnice prvního řádu}
    Nejprve se budeme věnovat řešení jedné diferenciální rovnice prvního řádu,
    \begin{equation}
        \derivative{y}{t}=f(y,t)
    \end{equation}
    s počáteční podmínkou
    \begin{equation}
        y(t_{0})=y_{0}.
    \end{equation}
    Zde $y=y(t)$ je hledaná funkce a $t$ je nezávisle proměnná.
    
    Numerické řešení diferenciální rovnice spočívá v nahrazení infinitezimálních přírůstků přírůstky konečnými:
    \begin{equation}\label{eq:Difference}
        \frac{\Delta y}{\Delta t}=\phi(y,t)
    \end{equation}
    kde $\phi$ je funkce, která udává směr, podél kterého se při numerickém řešení vydáme.
    Volbá této funkce je klíčová a záleží na ní, jak přesné řešení dostaneme a jak rychle ho dostaneme.

    \subsubsection{Pár důležitých pojmů}
    \begin{itemize}
        \item{\bf Explicitní algoritmy:}
        K výpočtu hodnoty funkce $y_{i+1}$ se vyžadují pouze hodnoty z aktuálních a minulých kroků, tj. $y_{i}$, $y_{i-1}$, atd.

        \item {\bf Jednokrokové algoritmy:}
        K výpočtu hodnoty funkce $y_{i+1}$ v následujícím kroku vyžadují pouze znalost hodnoty funkce v aktuálním kroku $y_{i}$.
        Rozepsáním~\eqref{eq:Difference} dostaneme
        \begin{equation}
            \boxed{
                y_{i+1}=y_{i}+\underbrace{\phi(y_{i},t)}_{\phi_{i}}\Delta t
            },
        \end{equation}
        přičemž počáteční hodnota $y_{0}$ je dána počáteční podmínkou.
        My se omezíme pouze na tyto algoritmy.

        \item {\bf Lokální diskretizační chyba:}
        \begin{equation}
            \mathcal{L}=y(t+\Delta t)-y(t)-\phi(y(t),t)\Delta t,
        \end{equation}        
        kde $y(t)$ udává přesné řešení v čase $t$.

        \item {\bf Akumulovaná diskretizační chyba:}
        \begin{equation}
            \epsilon_{i}=y_{i}-y(t_{i})
        \end{equation}

        \item {\bf Řád metody:} 
        Metoda je $p$-tého řádu, pokud
        \begin{equation}\label{eq:MethodOrder}
            L(\Delta t)=\mathcal{O}(\Delta t^{p+1}).
        \end{equation}

        \item {\bf Kontrola chyby řešení:}
        Chybu numerického řešení diferenciální rovnice lze zmenšit 1)~menším krokem, 2)~lepší metodou (metodou vyššího řádu). 
        Menší krok však znamená vyšší výpočetní čas.
        Sofistikované metody proto průběžně mění velikost kroku: když se funkce mění pomalu, krok prodlouží, když se mění rychle, krok zkrátí (tzv. {\bf metody s adaptivním krokem}).
        Tím se docílí vysoké přesnosti při co nejmenším výpočetním čase.

    \end{itemize}

    \subsubsection{Eulerova metoda 1. řádu}
        \begin{equation}\label{eq:Euler1}
            \phi_{i}=f(y_{i},t_{i}),
        \end{equation}
        tj. krok do $y_{i+1}$ děláme vždy ve směru tečny v bodě $y_{i}$.

        \begin{itemize}
            \item Nejjednodušší metoda integrace diferenciálních rovnic.
            \item Chyba je obrovská, k dosažení přesných hodnot je potřeba velmi malého kroku, což znamená dlouhý výpočetní čas.
        \end{itemize}

    \subsubsection{Eulerova metoda 2. řádu}
        \begin{align}\label{eq:Euler2a}
            k_{1}&=f(y_{i},t_{i})\nonumber\\
            k_{2}&=f\left(y_{i}+k_{1}\Delta t,t+\Delta t\right)\\
            \phi_{i}&=\frac{1}{2}\left(k_{1}+k_{2}\right),\nonumber
        \end{align}
        tj. uděláme jednoduchý Eulerův krok ve směru $k_{1}$, spočítáme derivaci $k_{2}$ po tomto kroku a vyrazíme z bodu $y_{i}$ ve směru, který je průměrem obou směrů (doporučuji si nakreslit obrázek).

        Ekvivalentní je udělat \uv{Eulerův půlkrok} a vyrazit z bodu $y_{i}$ ve směru derivace spočtené po~tomto půlkroku:
        \begin{align}\label{eq:Euler2b}
            k'_{1}&=f(y_{i},t_{i})\nonumber\\
            k'_{2}&=f\left(y_{i}+k'_{1}\frac{\Delta t}{2},t+\frac{\Delta t}{2}\right)\\
            \phi_{i}&=k'_{2}\nonumber
        \end{align}

    \subsection{Runge-Kuttova metoda 4. řádu}
        \begin{align}\label{eq:RungeKutta}
            k_{1}&=f(y_{i},t_{i})\nonumber\\
            k_{2}&=f\left(y_{i}+k_{1}\frac{\Delta t}{2},t+\frac{\Delta t}{2}\right)\nonumber\\
            k_{3}&=f\left(y_{i}+k_{2}\frac{\Delta t}{2},t+\frac{\Delta t}{2}\right)\\
            k_{4}&=f\left(y_{i}+k_{3}\Delta t,t+\Delta t\right)\nonumber\\
            \phi_{i}&=\frac{1}{6}\left(k_{1}+2k_{2}+2k_{3}+k_{4}\right)\nonumber
        \end{align}
        
        \begin{itemize}
            \item Jedna z nejčastěji používaných metod.
            \item Vysoká rychlost a přesnost při relativní jednoduchosti.
            \item Existují i Runge-Kuttovy metody vyššího řádu $p$, avšak vyžadují výpočet více než $p$ dílčích derivací $k_{j}$.
            Obecně platí, že metoda řádu $p\leq4$ vyžaduje $p$ derivací, metoda řádu $5\leq p\leq7$ vyžaduje $p+1$ derivací a metoda řádu $p=8,9$ vyřaduje $p+2$ derivací.
        \end{itemize}

    \begin{task}
        Naprogramujte Eulerovu metodu 1. a 2. řádu a Runge-Kuttovu metodu.
        Vyřešte diferenciální relaxační rovnici
        \begin{equation}\label{eq:Exp}
            \derivative{y}{t}=-y
        \end{equation}
        s počátečními podmínkami $y_{0}=1$ (analytickým řešením je funkce $e^{-t}$).
        Integrační krok $\Delta t$ ponechte jako volný parametr.
        Nakreslete grafy řešení $y(t)$ pro rozdílné hodnoty integračních kroků, například $\Delta t=0.01$ a $\Delta t=0.1$ pro čas $t\in\langle0;10\rangle$.
    \end{task}

    \begin{task}
        Rozšiřte kód tak, aby počítal průměrnou kumulovanou chybu
        \begin{equation}\label{eq:ExpError}
            \mathcal{E}=\sqrt{\frac{1}{n}\sum_{i=0}^{n-1}\left(y_{i}-e^{-t_{i}}\right)^{2}}
        \end{equation}
        a nakreslete závislost $\mathcal{E}(\Delta t)$ pro $\Delta t\in\langle0.002;0.1\rangle$ a pro různé metody.
        Jelikož očekáváme mocninnou závislost dle~\eqref{eq:MethodOrder}, kde exponent je tím větší, čím větší je řád metody, je výhodné graf $\mathcal{E}(\Delta t)$ kreslit v log-log měřítku.
        V Pythonu použijete místo \textnormal{\texttt{plot(...)}} funkci \textnormal{\texttt{loglog(...)}} z knihovny \textnormal{\texttt{matplotlib.pyplot}}.
        Ověřte, že získané křivky jsou v souladu s řády použitých metod.
    \end{task}

    \begin{task}
        Pomocí naprogramovaných metod vyřešte nelineární diferenciální rovnici
        \begin{equation}
            \derivative{y}{t}=\sin\left(t y\right)
        \end{equation}
        s počáteční podmínkou $y_{0}=1$ a vykreslete graf jejího řešení.
    \end{task}

\end{document}
