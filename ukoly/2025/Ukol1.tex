\documentclass[a4paper,11pt,twoside]{article}
\usepackage[utf8]{inputenc}	% Text coding
\usepackage[T1]{fontenc}
\usepackage{lmodern}
\usepackage[czech]{babel}
\usepackage{epsfig}
\usepackage{amsfonts,amsmath,amssymb}
\usepackage{graphicx}
\usepackage[unicode]{hyperref}
\usepackage{indentfirst}
\usepackage{fancyhdr}
\usepackage{xifthen}
\usepackage{amsthm,thmtools}
\usepackage{bold-extra}
\usepackage[dvipsnames]{xcolor}
\usepackage[subrefformat=simple,labelformat=simple]{subcaption} % Instead of subfigure
\usepackage{listings}
\usepackage{comment}
\usepackage{titlesec}
\usepackage{underscore}
\usepackage{makecell}       % Šířky čar v tabulkách

% Page size
\addtolength{\topmargin}{-1.5cm} %\addtolength{\textheight}{-10cm}
\addtolength{\textwidth}{4cm} \addtolength{\textheight}{4cm} % Width and height of the text
\addtolength{\voffset}{-0.5cm} % Top margin
\addtolength{\hoffset}{-2cm}
\setlength{\headheight}{15pt}

\DeclareMathOperator{\e}{e}

\def\vector#1{\boldsymbol{#1}}								% Vector
\renewcommand{\d}{\mathrm{d}}
\newcommand{\derivative}[3][]{\ifthenelse{\isempty{#1}}	    % Normal derivative
	{\frac{\d{#2}}{\d{#3}}}
	{\frac{\d^{#1}{#2}}{\d{#3}^{#1}}}
}
\newcommand{\im}{\mathrm{i}}

\def\makematrix#1{\begin{pmatrix}#1\end{pmatrix}}       % Matrix
\def\abs#1{\left|#1\right|}
\def\probability#1{\mathrm{Pr}\left[#1\right]}
\def\expectation#1{\mathrm{E}\left[#1\right]}
\def\dispersion#1{\sigma_{#1}^{2}}
\def\c{,\!}

\def\code#1{\textnormal{\texttt{#1}}}
\def\file#1{\textnormal{\textbf{\texttt{#1}}}}
\def\ghfile#1#2{\textnormal{\textbf{\texttt{\href{https://github.com/PavelStransky/PCInPhysics2021/blob/main/#1#2}{#2}}}}}

\def\abbreviation#1{\textnormal{\textsc{#1}}}

\begin{document}

\section*{Domácí úkol na 3.3.2025}
\subsection*{Logistické zobrazení}
\begin{enumerate}
    \item 
        Naprogramujte výpočet \emph{trajektorie} délky $N$ logistického zobrazení
        \begin{equation}
            x_{n+1}=f(x_{n})=ax_{n}(1-x_{n}),
        \end{equation}
        kde 
        \begin{itemize}
            \item $x_{0}\in(0,1)$ je počáteční podmínka,
            \item $x_{n}\in(0,1)$ je hodnota v diskrétním časovém kroku $n$,
            \item $a\in\langle 0,4\rangle$ je růstový parametr zobrazení.
        \end{itemize}
    \item
        Vykreslete graf trajektorie.

    \item 
        Vykreslete \emph{bifurkační diagram} logistického zobrazení na intervalu $a\in\langle a_{\mathrm{min}},a_{\mathrm{max}}\rangle$.
        Jedná se o množinu bodů, jejichž $x$-ová souřadnice je hodnota parametru $a$ a $y$-ová souřadnice jsou hodnoty $x_{n}$ pro $M\leq n\leq N$; parametr $M$ udává čas potřebný k relaxaci.

    \item
        Vypočítejte odhad \emph{Ljapunovova exponentu} pro logistické zobrazení a vykreslete ho do grafu na intervalu $a\in\langle a_{\mathrm{min}},a_{\mathrm{max}}\rangle$.
        Ljapunovův exponent je definován jako
        \begin{equation}
            \lambda=\lim_{n\to\infty}\frac{1}{n}\sum_{j=0}^{n-1}\ln\abs{f'(x_{j})},
        \end{equation}
        Limitu $n\rightarrow\infty$ aproximujte rozumně velkým číslem $N_\lambda$.

    \item
        Nalezněte numericky několik nejnižších \emph{bifurkačních bodů} $a^b_{j}$ na intervalu $a\in\langle 3,3.56994567\rangle$, tj. bodů, ve kterých se zdvojnásobuje perioda asymptotické trajektorie.
        První bifurkační bod je $a^b_{1}=3$.

    \item
        Z hodnot $a^b_{j}$ odhadněte \emph{Feigenbaumovu konstantu} $\delta$ definovanou jako
        \begin{equation}
            \delta=\lim_{j\to\infty}\frac{a^b_{j+1}-a^b_{j}}{a^b_{j+2}-a^b_{j+1}}.
        \end{equation}
\end{enumerate}

Vypracovaný úkol nahrajte do modulu Studijní mezivýsledky v SISu.
Před odevzdáním úkolu se přesvědčte, že program neobsahuje žádné syntaktické chyby a že je z kódu pochopitelné, jak ho spustit, aby vrátil hledaný výsledek.
Pokud řešení obsahuje více souborů, uložte je do jednoho souboru typu ZIP a nahrajte tento soubor.
\end{document}