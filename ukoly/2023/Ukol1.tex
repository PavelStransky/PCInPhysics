\documentclass[a4paper,11pt,twoside]{article}
\usepackage[utf8]{inputenc}	% Text coding
\usepackage[T1]{fontenc}
\usepackage{lmodern}
\usepackage[czech]{babel}
\usepackage{epsfig}
\usepackage{amsfonts,amsmath,amssymb}
\usepackage{graphicx}
\usepackage[unicode]{hyperref}
\usepackage{indentfirst}
\usepackage{fancyhdr}
\usepackage{xifthen}
\usepackage{amsthm,thmtools}
\usepackage{bold-extra}
\usepackage[dvipsnames]{xcolor}
\usepackage[subrefformat=simple,labelformat=simple]{subcaption} % Instead of subfigure
\usepackage{listings}
\usepackage{comment}
\usepackage{titlesec}
\usepackage{underscore}
\usepackage{makecell}       % Šířky čar v tabulkách

% Page size
\addtolength{\topmargin}{-1.5cm} %\addtolength{\textheight}{-10cm}
\addtolength{\textwidth}{4cm} \addtolength{\textheight}{4cm} % Width and height of the text
\addtolength{\voffset}{-0.5cm} % Top margin
\addtolength{\hoffset}{-2cm}
\setlength{\headheight}{15pt}

\DeclareMathOperator{\e}{e}

\def\vector#1{\boldsymbol{#1}}								% Vector
\renewcommand{\d}{\mathrm{d}}
\newcommand{\derivative}[3][]{\ifthenelse{\isempty{#1}}	    % Normal derivative
	{\frac{\d{#2}}{\d{#3}}}
	{\frac{\d^{#1}{#2}}{\d{#3}^{#1}}}
}
\newcommand{\im}{\mathrm{i}}

\def\makematrix#1{\begin{pmatrix}#1\end{pmatrix}}       % Matrix
\def\abs#1{\left|#1\right|}
\def\probability#1{\mathrm{Pr}\left[#1\right]}
\def\expectation#1{\mathrm{E}\left[#1\right]}
\def\dispersion#1{\sigma_{#1}^{2}}
\def\c{,\!}

\def\code#1{\textnormal{\texttt{#1}}}
\def\file#1{\textnormal{\textbf{\texttt{#1}}}}
\def\ghfile#1#2{\textnormal{\textbf{\texttt{\href{https://github.com/PavelStransky/PCInPhysics2021/blob/main/#1#2}{#2}}}}}

\def\abbreviation#1{\textnormal{\textsc{#1}}}

\begin{document}

\section*{Domácí úkol na 9.3.2022}
\subsection*{Tlumené a buzené kyvadlo}
Naprogramujte a odlaďte řešení pohybové rovnice pro pohyb tlumeného a buzeného kyvadla
\begin{equation}
    \label{eq:pendulum}
    \derivative[2]{\theta}{t}+\eta\derivative{\theta}{t}+\omega^{2}\sin{\theta}=A\sin\omega_{0}t,
\end{equation}
kde $\theta$ udává výchylku kyvadla od svislé osy, druhý člen popisuje tlumení v závěsu s koeficientem tlumení $\eta>0$, třetí člen odpovídá gravitační síle (a způsobuje nelinearitu rovnice) a pravá strana modeluje harmonické buzení s úhlovou frekvencí $\omega_0$.

Pro numerické řešení převeďte rovnici~\eqref{eq:pendulum} na soustavu obyčejných diferenciálních rovnic 1. řádu a k jejímu řešení použijte Eulerovu metodu 2. řádu a Runge-Kuttovu metodu 4. řádu:
\begin{align*}
    \vector{k}_{1}&=\vector{f}(\vector{y}_{i},t_{i}),\\
    \vector{k}_{2}&=\vector{f}\left(\vector{y}_{i}+\vector{k}_{1}\frac{\Delta t}{2},t_{i}+\frac{\Delta t}{2}\right),\\
    \vector{k}_{3}&=\vector{f}\left(\vector{y}_{i}+\vector{k}_{2}\frac{\Delta t}{2},t_{i}+\frac{\Delta t}{2}\right),\\
    \vector{k}_{4}&=\vector{f}\left(\vector{y}_{i}+\vector{k}_{3}\Delta t,t_{i}+\Delta t\right),\\
    \vector{\phi}_{i}&=\frac{1}{6}\left(\vector{k}_{1}+2\vector{k}_{2}+2\vector{k}_{3}+\vector{k}_{4}\right),
\end{align*}
kde $\vector{y}$ je vektor hledaných funkcí, $\vector{f}$ je vektor pravé strany soustavy řešených diferenciálních rovnic prvního řádu a $\Delta t$ je časový krok.
Integrační krok se dělá stejně jako u ostatních procvičovaných jednokrokových algoritmů,
\begin{equation*}
    \vector{y}_{i+1}=\vector{y}_{i}+\vector{\phi}_{i}\Delta t.
\end{equation*}

Uvažujte následující hodnoty parametrů
\begin{align*}
    \eta&=0.5,
    &\omega&=1,
    &A&=1.
\end{align*}
Pro frekvenci buzení volte tři různé hodnoty $\omega_0=0\c2;0\c4;0\c8$ a řešte na časovém intervalu $t=\langle0;500\rangle$ s počátečními podmínkami $\theta_0=0,\dot{\theta}_{0}=1$.
Otestujte stabilitu řešení volbou dvou různých časových kroků $\Delta t=0\c01$ a $\Delta t=0\c005$.

Vypracovaný úkol odešlete na e-mailovou adresu \href{mailto:pcfyzika@pavelstransky.cz}{pcfyzika@pavelstransky.cz}.
Před odesláním se přesvědčte, že program neobsahuje žádné syntaktické chyby a že je z kódu pochopitelné, jak ho spustit, aby vrátil hledaný výsledek.
\end{document}